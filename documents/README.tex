\documentclass[10pt]{jsarticle}
\title{電通大剣道部 新規HP作成計画}

\begin{document}
\maketitle


\section{載せるコンテンツ一覧}
\begin{itemize}
	\item トップページ
	\item 部活紹介・指導陣
	\item 予定表
	\item 部員紹介
	\item リンク
	\item 試合結果
	\item 活動風景
\end{itemize}


\section{各種コンテンツ作成手法}

\subsection{静的HTML}
HTMLでファイルを作り、それを設置するのみ。
編集時には直接HTMLを記述する必要があるため、更新頻度の高いページには不向き。
\begin{itemize}
	\item 部活紹介・指導陣 
	\item リンク
	\item 予定表
\end{itemize}

\subsection{CMS(WordPressなど)}
ブログのように更新頻度の高いページの管理に便利。
\begin{itemize}
	\item トップページ(お知らせとか)
	\item 活動風景(Facebookで十分?)
\end{itemize}

\subsection{自動生成}
データベースとの連携を取りたいページは自動生成する。
どのようなフレームワークで作るかは未定。
なお、別途管理機能の実装が必要。(SQL文打ちたくない)
\begin{itemize}
	\item 部員紹介
	\item 試合結果
\end{itemize}
\end{document}
